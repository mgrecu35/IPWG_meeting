\documentclass[final,t]{beamer}
\mode<presentation>
  {
    \usetheme{CambridgeUS}
    \usecolortheme{seahorse}
  }

\usepackage[orientation=landscape,size=a1,scale=1,debug]{beamerposter}
\usepackage{lipsum} % for dummy text
\usepackage{graphicx}
\setbeamertemplate{caption}[numbered]

\title[]{\huge PSD Parameterization to support ice retrievals from multiple
instrument observations}
\author[]{\large Mircea Grecu$^{1}$, Gerald Heymsfield$^{2}$, John Yorks$^{2}$, and Andrew Heymsfield$^3$}
\institute[]{\Large (1) Morgan State University,  (2) NASA GSFC, and (3) NCAR}
\date{}

\begin{document}

\begin{frame}
\maketitle
\begin{columns}[]
  \begin{column}{.32\linewidth}

\begin{block}{Motivation}
\begin{itemize}
\item
  Current and future air- and space-borne observing systems feature
  instruments with different sensitivities and subject to attenuation at
  different ranges.
\item
  From the scientific perspective, physically consistent ice property
  estimates are desirable irrespective of the instruments used in their
  derivation.
\item Such estimates may be derived through the development of consistent "a priori"
probability distributions.
\end{itemize}
\end{block}

\begin{block}{Approach}
\begin{itemize}
\item  Use "in-situ" Particle Size Distributions (PSDs) to simulate lidar and radar observations at X-, Ku-,Ka- and W- band.
\item  Develop kNN methodology to estimate ice properties from combination of observations.
\item Test methodology using cross-validation.
\item Evaluate using independent estimates.
\end{itemize}
\end{block}

\begin{table}[ht]
%\caption{A table arranging  images}
\centering
\begin{tabular}{cc}
\includegraphics[scale=0.85]{IMPACTS_Nw.png}&\includegraphics[scale=0.85]{IMPACTS_Nw_dm.png}\\

%\multicolumn {2}{c} 
\end{tabular}
\end{table}
\centerline{The $N_w$ distribution as a function of temperature and the $N_w-D_m$ joint
distribution}

\centerline{\includegraphics[scale=0.95]{radarAndLidar_slice.png}}
\centerline{Simulated lidar backscatter and radar reflectivitiy.}

\begin{itemize}
\item
In the combined radar lidar region, the lidar backscatter and the integrated backscatter can be used
as constraints for the kNN methodology.
\item
In the region of single frequency radar observations, temperature can be directly used as a parameter,
over via $N_w$ in "normalized relationships" e.g. $IWC=N_w^{1-b}aZ_w^b$.
\end{itemize}
\end{column}


 %%%%%%%%%%%%%%%%%%%%%%%%%%%%%%%%%%%%%%%%%%%%%%%%%%%%%%%%%%%


  \begin{column}{.32\linewidth}

  \begin{block}{Cross-Validation Results}
  \begin{itemize}
   \item
   For evaluation in the simulation space, the database of observed PSDs and calculated reflectivities is randomly split in two.
   \item
   Half of the database is used for training, while the other is used for evaluation.
   \item
   Results are shown below.  Errors are likely to be caused by ambiguities in the database, rather than
   methodology.
  \end{itemize}
  \end{block}
\begin{table}[ht]
%\caption{A table arranging  images}
\centering
\begin{tabular}{cc}
\includegraphics[scale=1]{dmValidation.png}&\includegraphics[scale=1]{iwcValidation.png}\\
\end{tabular}
\end{table}

\begin{block}{Direct Validation Results}
\begin{itemize}
\item
 "In-situ" and retrieved IWCs for two cases from the 2020 campaign are shown below.
 \end{itemize}
\end{block}
  \centerline{\includegraphics[scale=1.25]{jan25_2020.png}}
  
  \centerline{\includegraphics[scale=1.25]{feb05_2020_2.png}}
  
%\begin{table}[ht]
%\caption{A table arranging  images}
%\centering
%\begin{tabular}{cc}
%\includegraphics[scale=0.75]{jan25_2020.png}&\includegraphics[scale=0.75]{feb05_2020_2.png}\\
%\end{tabular}
%\end{table}
\end{column}


  %%%%%%%%%%%%%%%%%%%%%%%%%%%%%%%%%%%%%%%%%%%%%%%%%%%%%%%%%%%

  \begin{column}{.32\linewidth}

  \begin{block}{Conclusions}
   And here I am going to conclude.
  \end{block}
  
   \centerline{\includegraphics[scale=1.25]{retrievedProfs.png}}
  \end{column}

  \end{columns}

\end{frame}

\end{document}
